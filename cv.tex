\documentclass[12pt]{article}
\begin{document}
SUMMARY

I'm an enthusiastic professional with more than 10 years of software development experience. I love learning new things and solving interesting problems. I have a Masters degree in Bioinformatics and a passion for machine learning, artificial intelligence, and everything related.
EXPERIENCE

GOCARDLESS, Latvia, Riga
Duration: 01.2024 – Present

Project
Working on supporting and expanding open banking offering
Duration
01.2024 – Present
Role
Software Developer
Technologies
Python, Django, Typescript, React, Node.js, Ruby, Postgres, Docker, Kubernetes, Kibana, Grafana, Prometheus, Git
Description
I’m working on expanding, improving, and adapting current open banking solutions to better suit the needs of the business.

SANNSYN AS, Latvia, Riga
Duration: 09.2019 – 01.2024

Project
Consultancy work for a customer in insurance and asset management
Duration
11.2020 – 01.2024
Role
Data Engineer / Developer
Technologies
C sharp, Python, PySpark, Node.js, SQL, Microsoft SQL, Db2, Microsoft Synapse, Databricks, ADLS, dbt, Snowflake, Power BI, Git
Description
I helped a customer maintain existing legacy systems, and assisted in the transition from the current systems to a new solution. During this time I:
expanded the customer's current mathematical models with new rules and implemented new formulas while closely cooperating with actuaries,
fixed issues in the models they have,
created a basic C sharp crawler in Python to help with the effort of documenting the current legacy modules and performed manual analysis of these modules,
created migration scripts that achieve a 100x increase in speed over the code that they're replacing,
worked on creating and populating a new data warehouse solution.

Project
Data aggregation for a customer in retail
Duration
08.2020 – 10.2020
Role
Data Scientist
Technologies
Python, PySpark, Apache Spark, PostgreSQL, Apache Cassandra, Git
Description
I set up a Spark cluster and developed a set of scripts for aggregating impression and click data in a consumable form.

Project
Consultancy work for a customer in retail
Duration
11.2019 – 05.2020
Role
Data Scientist
Technologies
Python, PySpark, scikit-learn, Microsoft Azure, Git
Description
I worked on extracting sales insights from large amounts of transactional data.

Project
Consultancy work for a customer in publishing
Duration
09.2019 – 09.2020
Role
Data Scientist
Technologies
Python, Gensim, NLTK, Git
Description
I processed a proprietary set of Norwegian synonym data and evaluated the impact that could be achieved by enriching a Word2Vec based semantic system with this data. Once the evaluation was done, I added a configurable extension to the semantic system that allowed for the use of synonym data when available.

Project
TellusR NLP
Duration
09.2019 – 05.2021
Role
Data Scientist / Developer
Technologies
Python, Kotlin, Solr, Gensim, NLTK, spaCy, Django, nmslib, Git
Description
I worked on a Word2Vec and Doc2Vec driven semantic engine with the goal of improving the product relevance of retrieved by e-commerce site search engines. During the development process I identified suitable hyperparameters for Latvian, German, English, and Norwegian embeddings and improved the engine's responsiveness, achieving a 10x reduction in response time and 4x reduction in memory footprint without deteriorating the models' accuracy.

Project
Loyalty Engine
Duration
09.2019 – 10.2019
Role
Developer
Technologies
Javascript, OpenUI5, Node.js, Git
Description
I worked on implementing the front-end of a rule based loyalty engine.

UNIVERSITY OF LATVIA, Latvia, Riga
Duration: 03.2020 – 12.2020

Project
Metagenomics analysis
Duration
03.2020 – 12.2020
Role
Research Assistant
Technologies
Python, Slurm, Bash, Snakemake, StaG-mwc, Git
Description
I made bacterial and antibiotic resistome reference databases and created pipelines that allowed for the classification of more than 24 TB of sample data with minimal manual effort on the institute's scientists' part.

EXIGEN SERVICES LATVIA, Latvia, Riga / EMERGN LIMITED, Latvia, Riga
Duration: 02.2012 – 08.2019

Project
Machine Learning Lab
Duration
11.2017 – 08.2019
Role
Lead Machine Learning Developer / Lead Data Scientist
Technologies
Python, Php, Javascript, jQuery, Flask, Django, Redis, MySql, OpenCV, Tensorflow, Keras, Gensim, NLTK, spaCy, PowerBI, Pandas, scikit-learn, Git, NetworkX, igraph
Description
I took part in establishing a machine learning focused team within the company and lead the development of machine learning based prototypes and tools including:
a tool for identifying key persons within an organization based on communications graphs,
a tool for the clustering and tagging of survey responses,
prototypes for predicting temperature fluctuations within buildings,
prototypes for anomaly detection in a large set of transactional data,
multiple motion tracking game prototypes for trade shows,
participated in discussions over the company's AI strategy.

Project
Travel sites
Duration
06.2014 – 11.2017
Role
Developer
Technologies
Php, Javascript, jQuery, Mustache, Slim, MySql, Redis, Mercurial, CSS, Sass
Description
I took part in the creation and maintenance of a number of travel websites. During this period we developed a custom framework to allow for rapid modernisation of all of the customers' sites, and I developed wishlist, comparison, and sorting modules that allowed for a better user experience when looking for flights or hotels. I also took part in developing a robust distributed data storage solution that allowed us to use a cluster of distributed Redis instances.

Project
App Gallery
Duration
02.2012 – 06.2014
Role
Developer
Technologies
Java, Hibernate, Javascript, Richfaces, JSF, JBoss, CSS, SQL, Svn
Description
I worked on maintaining a customer's in-house app store.

EDUCATION

Uppsala University, Master of Science in Bioinformatics, 2015 – 2017
Thesis on the “Local adaptation of Grauer’s gorilla gut microbiome”, which involved processing short read sequencing data with bioinformatics tools, the creation of pipelines that would enable the analysis of results produced by these tools, and the analysis of the obtained results.
Riga Technical University, Bachelor of Computer Science, 2010 – 2013
Thesis on the “Usage of genetic algorithms in the construction of decision trees”, which involved the analysis of a biological data set, the classification of its entries with machine learning algorithms, and the analysis of the obtained results.

\end{document}
